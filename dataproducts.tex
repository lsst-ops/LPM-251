\section{Types of Data Products for iDACs}\label{sec:data}

 {\color{red} In quite good shape, Leanne to flesh out with demonstrative science use cases and experience from other archives} \newline

The full collection of LSST data products, from images to catalogs, is described by the Data Products Definitions Document \citedsp{LSE-163}. Below, three potential levels of data products that iDACs might consider hosting are described: the full data release with images, the data release catalogs, and a low-volume (``{\tt lite}'') subset of the catalogs.

\subsection{Full Release(s)}

In this case the iDAC would be hosting all of the raw and processed images and catalogs, as described in \citedsp{LSE-163}. Including the raw image data in an iDAC requires roughly $6$ petabytes per year of survey, so this is a serious augmentation on resources in terms of both hardware and personnel. The processed data and associated calibrations bring the total data volume to $0.5$ exabytes for a single data release. Some data volume could be saved by taking only a single calibrated image per band, but the total would still be $60$ petabytes (with compression it may be possible to reduce this even further). Any iDAC considering hosting the full data release should also deploy the full LSST Science Platform \citeds{LSE-319} in order to maximize science productivity and their return on investment in hosting an iDAC.

\subsection{Catalog Server}

Alternatively, an iDAC may find that hosting only the data release catalogs, and not the images, is sufficient for the scientific needs of its community. This will probably require the specific LSST database server \citedsp{LDM-135} and specific machines, and the deployment of the database system and the associated subset of data access services (DAX; e.g., web APIs, Qserv, \citeds{LDM-152}). The full {\tt Object} catalog, which contains one row per object with a volume of $\approx 20$ kilobytes per row, is estimated to contain about $40 \times 10^9$ objects (even in the first full-sky data release). Adding to this the full {\tt Source} and {\tt Forced Source} catalogs, which contains one row per measurement in each of the $\sim80$ visit images obtained per year, brings the total storage volume required up into the petabytes range, and will require a serious commitment of resources at the proposed iDAC. The evolution of catalog sizes over the 10-year LSST survey is depicted in \figref{fig:catvol}, from which it is evident that the catalog size for the final release is order $15$ petabytes. For more details on the row counts see the Key Numbers Page\footnote{\url{https://confluence.lsstcorp.org/display/LKB/LSST+Key+Numbers}}.

\begin{figure}
\begin{center}
\includegraphics[width=0.8\textwidth]{CatVolTime.png}
\caption{Catalog volume over time from \citeds{LDM-144}. \label{fig:catvol}}
\end{center}
\end{figure}

\subsection{An ``{\tt Object Lite}" Catalog}\label{sec:lite}

Many -- perhaps most -- astronomers' science goals will be adequately served by a low-volume subset of the {\tt Object} catalog's columns that do not include, for example, the full posteriors for the bulge+disk likelihood parameters. This {\tt Object Lite} catalog would nominally contain $1840$ bytes per row for the $40 \times 10^{9}$ objects, giving a size of $\approx 7.4 \times 10^{13}$ bytes ($\sim74$ terabytes). Even smaller, science-specific versions of {\tt Object Lite} could be envisioned, with even less columns and/or separate star and galaxy catalogs.
These would not be small enough to handle on a laptop, but might be served by a small departmental cluster (a mini-iDAC). Searching even a small {\tt Object Lite} catalog would require some form of database, but many institutes would already have a system which may be capable of loading this data. In this case, LSST might only ship files with documentation and not provide administrative support for the system, but this would allow the {\tt Object Lite} catalog to be widely available to all partner institution iDACs. Distribution options such as peer-to-peer networking to avoid download bandwidth limitations might be possible to implement in this case.i See also \secref{sec:public}.

