\section{Introduction}\label{sec:intro}


In 2019 LSST with NSF and DOE has considered that maximizing science is best achieved through an Open Data Framework (ODF).
Getting data to as many scientist as possible without restrictions will foster more collaborations and should give maximum science return. Gaia's \citep{2016A&A...595A...1G} open data policy has certainly shown this to be the case with around one thousand publications since the first data release. SDSS also had a open data policy and has over eight thousand publications since 2003 (DR1).
% not very well done ADS search restricted to staert from year 2015 for Gaia and 2003 for SDSS - number of refereed papers.
This is a departure from the previous LSST notion of proprietary data products.


 Access to LSST data products for any users will be possible through a Data Access Center (DAC). The United States's DAC will be hosted at the National Center for Supercomputing Applications (NCSA),
where authorized LSST users will perform scientific queries and those with sufficient privileges will be allowed to perform analysis on the full data releases using the LSST Science Platform (LSP). The LSP is now well documented with the vision given in \citeds{LSE-319}, with more formal requirements in \citeds{LDM-554} and the design in \citeds{LDM-542}. The Chilean DAC will be equivalent in functionality to the US DAC, but scaled-down in terms of the computational resources available for query and analysis given the smaller Chilean community \citedsp{LDM-572}\footnote{Most recently the Chilean Government has started the data observatory initiative \url{https://www.economia.gob.cl/data-observatory} which may see this DAC moved to the cloud}.

Within the ODF data products may be accessed from a host of other potential locations where one would normally expect to find astronomical data. This is discussed further in \secref{sec:public}


%This document proposes a set of guidelines and policies for partner institutions -- in the US, Chile, or one of the International Contributors with signed Memoranda of Agreement -- that are interested in hosting the LSST data, in whole or in part, for their affiliated members as an independent Data Access Center (iDAC).
The following sections include the types of data products that could be hosted (Section \ref{sec:data}), the requirements and responsibilities that would be expected of an iDAC hosting LSST proprietary data products (Section \ref{sec:reqs}), and a description of the main costs {\it vs.} their science impacts (Section \ref{sec:costs}).

The contents of this draft are meant to provide a preliminary resource for partner institutions who may be assessing the feasibility of hosting an iDAC. The specific mechanisms and processes by which future iDACs will negotiate the bulk transfer of data, the installation of software, etc. is considered beyond the scope of this document.

To help get an idea of sizes \tabref{tab:sizes} gives an overview.


\begin{table}
\caption{Size summary based on \citeds{LDM-141} \label{tab:sizes}}

\begin{tabular}{l r l r r r }
\hline
 \bf{Table}  &    \bf{Bytes/row} &\bf{Rows (DR1 -> DR11)} &\bf{DR1 (TB)} &\bf{ $\times$ Growth } &\bf{DR10 (PB)}\\
\hline
 Object\_Lite    &1840   &$2.26^{10} -> 4.74^{10}$ &42  &2.1  &0.08 \\
 Object\_Extra   &20393  &$2.26^{10} -> 4.74^{10}$ &461 &2.1  &0.9  \\
 Source          &453    &$4.51^{11} -> 9.01^{12}$ &204  &20.0 &4.0  \\
 ForcedSrc       &41     &$1.20^{12} -> 5.01^{13}$ &49  &42  &2.0  \\
 DiaObject       &1405   &$7.94^{08} -> 1.54^{10}$  &1.1  &19.4 &0.002  \\
 DiaSource       &417    &$2.26^{09} -> 4.52^{10}$  &0.9&20 &0.002 \\
 DiaForcedSource &49     &$1.50^{10} -> 3.01^{11}$  &0.7 &20  &0.001 \\
\hline
 \multicolumn{6}{l}{Year ~1   raw images:$ 3 PB$, tables:$\sim 1 PB$, half for Object\_Extra,$ 0.2 PB$  Sources}\\
 \multicolumn{6}{l}{Year 10   raw images:$ 30 PB$, tables:$\sim 7 PB$,$ 4 PB$  Sources,$ 2.0 PB$  Forced ,$ 1 PB$  Object\_Extra}\\
\hline
\end{tabular}
\end{table}


All access to, and use of, the LSST data and data products is subject to the policies described in \citeds{LPM-261}.
