\documentclass[LPM,obsolete]{lsstdoc}
\usepackage{enumitem}

\setDocChangeRecord{%
\addtohist{}{2018-03-24}{Initial version.} {WOM}
\addtohist{}{2019-02-25}{Added site topology schematic} {LPG}
\addtohist{}{2019-06-17}{In kind data framework } {WOM}
\addtohist{}{2020-04-01}{CEC input } {WOM}
%\addtohist{1.0}{2017-12-12}{Approved in \jira{RFC-xxx}.}{W.~O'Mullane, T.~Jenness}
}

\title[Independent DACs]{Proposed Policy for Independent Data Access Centers}

\author   {William O'Mullane, Beth Willman, Melissa Graham, Leanne Guy, Robert Blum }
\setDocRef      {LPM-251} % the reference code
\setDocDate     {\today}              % the date of the issue
\setDocUpstreamLocation{\url{https://github.com/lsst/LPM-251}}
%
% a short abstract
%
\setDocAbstract {
	See rather \url{RTN-003.lsst.io} This document describes the proposed policies for groups that are independent from the LSST Project and Operations (i.e. LSST Data Facility) and would like to stand up an independent Data Access Center (IDAC; existing data centers that could serve LSST data products are considered IDACs for purposes of this document). Some IDACs may want to serve only a subset of the LSST data products: this document proposes three portion sizes, from full releases to a "light" catalog without posteriors. Guidelines and requirements for IDACs in terms of data storage, computational resources, dedicated personnel, and user authentication are described, as well as a preliminary assessment of the cost impacts. Some institutions, even those inside the US and Chile, may serve LSST data products locally to their research community. Requirements and responsibilities for such institutional bulk data transfers are also described here. {\bf The purpose of this draft document is to serve as a {\it preliminary} resource for partner institutions wishing to assess the feasibility of hosting an IDAC.}
}
\graphicspath{ {./figs} {./images} }

\begin{document}
%
% the title page
%
\maketitle

\renewcommand{\thepage}{\arabic{page}}% Arabic numerals for page counter

\setcounter{page}{1}% Start page number
%\printnoidxglossaries
%
% It's all yours from here on
%
\section{Introduction}\label{sec:intro}

In 2019 LSST NSF and DOE  changed their stance on in kind contributions. This means LSST data products will be potentially  available to  a larger  science community in return for in kind contributions. LSST data products will be made available through the LSST Data Access Centers (DACs) in the US and Chile.
Getting data to as many scientists as possible\footnote{preferably without restrictions} will foster more collaborations and should give maximum science return. Gaia's \citep{2016A&A...595A...1G} open data policy has certainly shown this to be the case with around one thousand publications since the first data release. %SDSS also had a open data policy and has over eight thousand publications since 2003 (DR1). [this statement could confuse people since all SDSS data are first proprietary.
% not very well done ADS search restricted to start from year 2015 for Gaia and 2003 for SDSS - number of refereed papers.

%This is a departure from the previous LSST notion of proprietary data products.

Access to LSST data products for any users will be possible through a Data Access Center (DAC). The United States's DAC will be hosted at the National Center for Supercomputing Applications (NCSA),
where registered LSST users will perform scientific queries. Most users will have access to a default set of resources at the DAC sufficient for basic queries and analysis. Users who require more resources will be able to apply for them, and those granted additional resources will be allowed (for example) to perform analysis on the full data releases using the LSST Science Platform (LSP). The LSP is documented with the vision given in \citeds{LSE-319}, with more formal requirements in \citeds{LDM-554} and the design in \citeds{LDM-542}. The Chilean DAC will be equivalent in functionality to the US DAC, but scaled-down in terms of the computational resources available for query and analysis given the smaller Chilean community \citedsp{LDM-572}\footnote{Most recently the Chilean Government has started the data observatory initiative \url{https://www.economia.gob.cl/data-observatory} which could see this DAC or other access points moved to the cloud under ODF. This is TBD and outside scope for LSST}.
Within the ODF data products may be accessed from a host of other potential locations where one would normally expect to find astronomical data. This is discussed further in \secref{sec:public}

%This document proposes a set of guidelines and policies for partner institutions -- in the US, Chile, or one of the International Contributors with signed Memoranda of Agreement -- that are interested in hosting the LSST data, in whole or in part, for their affiliated members as an independent Data Access Center (iDAC).
The following sections include the types of data products that could be hosted (Section \ref{sec:data}), the requirements and responsibilities that would be expected of an iDAC hosting LSST proprietary data products (Section \ref{sec:reqs}), and a description of the main costs {\it vs.} their science impacts (Section \ref{sec:costs}).

The contents of this draft document are meant to provide a preliminary resource for partner institutions who may be assessing the feasibility of hosting an iDAC. The specific mechanisms and processes by which future iDACs will negotiate the bulk transfer of data, the installation of software, etc. is considered beyond the scope of this document.

To better understand the sizes of LSST data products, \tabref{tab:sizes} gives an overview.


\begin{table}
\caption{Size summary based on \citeds{LDM-141} \label{tab:sizes}}

\begin{tabular}{l r l r r r}
\hline
 \bftext{Table}  &    Bytes/row &Rows (DR1 -> DR11) &Total DR1 (TB) &Growth factor  DR10 (PB)
\hline
 Object_Lite &1840 &2.26E10 -> 4.74E10 &42  &2.1  &0.08 \\
 Object_Extra &20393  &2.26E10 -> 4.74E10 &461 &2.1  &0.9  \\
 Source &453 &4.51E11 -> 9.01E12 &204  &20.0 &4.0  \\
 ForcedSrc       &41 &1.20E12 -> 5.01E13 &49  &42  &2.0  \\
 DiaObject       &1405 &7.94E08 -> 1.54E10  &1.1  &19.4 &0.002  \\
 DiaSource       &417 &2.26E09 -> 4.52E10  &0.9&20 &0.002 \\
 DiaForcedSource &49 &1.50E10 -> 3.01E11  &0.7 &20  &0.001 \\
\hline
 Year ~1 &    raw images: 3 PB, tables: ~ 1 PB, half is for Object_Extra, 0.2 PB is Sources\\
 Year 10 &   raw images: 30 PB, tables: ~ 7 PB, 4 PB for Sources, 2.0 PB for Forced Sources,
            1 PB for Object_Extra\\
\hline
\end{tabular}
\end{table}


All access to, and use of the LSST data and data products is subject to the policies described in \citeds{LPM-261}.

%

\section{Public Data - the Open Data Framework}\labe{sec:public}


WIth public data be we face a potential huge number of users.
Alex Szalay at AAS noted there were over a million unique IP addresses which hit the SDSS archive over a 1 year period.  Gaia saw 2K users accounts per hour on the catalogue release - we are clearly not only serving the 10K professional astronomers in the world anymore.
This could be mitigated in two ways.


Nationally we could partner with Google or Amazon who will host public data sets for free - so we could more selectively add users to the DACs and try to put most casual users to the public interface. It is not clear if this would then be the EPO interface but that is worth considering i.e. EPO would no longer have to select 10\% of data but it would have a much bigger job to deal with all of it perhaps.
The public data set would be some version for the catalogue and the HIPS type color images. No raw data files and potentially no advanced notebook type access. This may not even need a logon for the trivial stuff like show me M31 with LSST sources plotted.  So no source catalogue which is the big one (10^13 rows)
Potential national partners  could also host the object catalogue e.g.
* MAST at STScI
* DataLab at NOAO
* SAO
* Naval Observatory ?
* NED at IPAC ?
* HEASARC
* CADC - Canadian Astronomical Data Centre - we work with them already in DM.

International
Internationally we could partner with s network of sites - it should be a network to allow peer to peer sharing of catalogues. Here we could provide the HIPS color images and again some version of the OBJECT catalogue. We would have to consider if the source catalogue should also be distributed to e smaller subset of centres who could cope with it.
Potential International Partners
* In Europe we have a few centres that Astronomers expect to find sources at:
    * CDS  - Aladain and Vizier - this is a minimum for Europe
    * ESAC - ESASky - European Space Agency
    * ASDC - Itiallian Data Center
    * GAVO MPA - German Astronomical Virtual Observatory - Max Plank Astrophysics
    * IOA - Cambridge
    * Edinburogh
* In Asia
    * NAOJ JADC- Tokyo - Given the HSC connection and the fact that Kazuhiro Sekiguchi approached at AAS with interest in joining LSST we shoudl talk to them
    * Some place in Australia - though they have been non committal
    * China ?
P


\input{dataproducts}
% LPG Detail here the minimum resources and levels of service that a iDAC must commit to to be qualify for consideration as an LSST iDAC
% I wouldn't call this guidelines - there are concrete requirements that must be met

\section{Requirements and Guidelines for iDACs}\label{sec:reqs}
Since creating, delivering, and supporting the implementation of LSST data products in iDACs creates some cost to the LSST Project, iDACs will be expected to follow some basic requirements and guidelines that are described below.
The actual costs of iDAC support and infrastructure development are considered separately in Section \ref{sec:costs}.



\subsection{LSST site topology} \label{sec:topology}

{\color{red}Leanne to describe the flow in the topology diagram.} \newline

\begin{figure}
\begin{center}
\includegraphics[width=0.8\textwidth]{images_local/LSST-site-topology}
\caption{LSST Data Facility and Independent DAC network topology.  \label{fig:lsst-site-topology}}
\end{center}
\end{figure}



\subsection{Resources}
{\color{red}Wil } \newline

\subsubsection{Data Storage}
Any institution considering setting up an iDAC will need to show commitment on purchasing sufficient storage and CPU power to hold and serve the data. Sufficient storage ranges from $0.5$ exabytes for the full data release(s) down to $100$ terabytes for a catalog server, and potentially further down to $70$ terabytes if the {\tt Object Lite} option is offered. For the full catalog it is order 100 nodes to serve it up, and to serve images a DAC would need some additional servers; depending on load this may be order 10 additional nodes.

\subsubsection{User Computational}
If the full set of data release products including images and catalogs are desired, it is highly recommended that the iDAC deploy the LSST Science Platform (LSP). The LSP serves as a portal to the data, and provides a user interface of web services and Jupyter notebooks for scientific queries and analysis, an open software framework for astronomy and parallel processing, and the associated computational, storage, and communications infrastructure needed to enable science. The LSP is described in full in \citeds{LSE-319} and \citeds{LDM-554}. Depending on the assumed load, the LSP is relatively modest as it requires only $\sim2$ servers to set up, and it is recommended to have 2 CPUs per simultaneous user (e.g., if the iDAC's desired capability is to serve 200 users, but only expect 50 to be active at a time, then 100 CPUs would be sufficient). From that starting point, the amount of next-to-the-data computational resources can be as large as the data center wishes to provide, and may make use of connecting to e.g., local super computer resources.

\subsubsection{Dedicated Personnel}
The significant hardware required by an iDAC is above the normal level for most astronomy departments, and would require dedicated technical personnel to set it up and keep it running. For an {\tt Object Lite} catalog running on existing hardware, this might not be a significant increase in person power if the hardware is already serving on order $50$--$100$ terabytes. Still, it is recommended to assume $\gtrsim0.25$ full-time equivalent (FTE) personnel hours for {\tt Object Lite}, and perhaps closer to $\sim2$ FTE for the full catalogs, which includes setting up and maintaining the service, and installing new data releases and software updates every year. For iDACs wishing to host the full data releases' images and catalogs and deploy the LSST Science Platform, it becomes necessary to employ $1$--$2$ storage engineers to mange the large amount of data, and possibly one more FTE to keep the Kubernetes (or equivalent) system updated with the latest software deploys. If the iDAC intends to support the science of many local users, support will become a specific issue which may not be covered by the usual institutional funding, and will require further effort. It is therefore recommended that any partner institution wishing to host a full-release iDAC provide a minimum personnel of 5 FTE to be considered viable.

\subsection{Services}
{\color{red} Leanne } \newline

Services provided by LSST partner iDACs:s

\subsubsection{The LSST Science Platform}
All LSST independent data access centres must run an instance of the LSST Science Platform

\subsubsection{User Generated data products }
% Detail here the responsibilities and commitments of the LDF to LSST iDACs. See annex 3.1

% LPG Outline the responsibilities of the LDF towards recognized iDACs
\section{Responsibilities of the LSST Data Facility}
{\color{red} Bob } \newline

This section describes the services that the LSST Data Facility (LDF) will provide in support of all LSST iDACs.

Points to be addressed
\begin{enumerate}
\item High-bandwidth network out of the LDF to all iDACs
\item Distribution of each of the data products types outlined in \ref{sec:data} to iDACs
\item ......
\end{enumerate}


All LSST  iDACs must be able to serve the 'object-lite' catalog to any institution worldwide.


\subsection{Proprietary Data Access Policies}
{\color{red}Defer for now until further guidance received.} \newline

All iDACs serving the proprietary LSST data products are subject to the policies in \citeds{LPM-261}. It might be necessary for iDACs to use the LSST authentication system to ensure secure access to the data for authorized users only, and this might require some specific IT work at the iDAC host site.

%%%MLG: commented this out (WOM, delete if we'll never need to say this).
% It remains an open question whether {\em any} qualified user from any LSST partner institution can log into any iDAC -- this would facilitate collaboration but would also require that iDACs participate in the single authentication system.

%%%MLG: commented this out (WOM, delete if we'll never need to say this).
%\subsection{iDACs Serving Post-Proprietary Data}
% {\bf MLG: can we think of any requirements/guidelines for this scenario? Should this document even cover non-partner iDACs? Any non-partner iDAC wishing to serve the two-year-old post-proprietary full release to its users would not, for example, be able to get help installing the LSP. Perhaps we could use this paragraph to further encourage the benefits of partnership.}

\section{Cost Impacts}\label{sec:costs}

{\color{red} Wil } \newline

As previously mentioned, standing up and maintaining multiple iDACs comes at a significant cost impact to both the LSST Project and the partner institutions. Minimizing these costs -- or at least maximizing the amount of science they enable -- should be at the forefront of all considerations concerning partner iDACs, such as the following propositions.

\subsection{Maximizing Profits with Science-Driven iDACs}
There are two main cost impacts of iDACs being set up outside of the US and Chilean DACs: the positive impact is that some computational load may be taken off of these existing DACs, but the negative impact is the level of support required from the LSST Project in order to get them set up and running. This negative impact could be mitigated by ensuring that science productivity is maximized as a result of this extended effort. One way to do this might be to associate specific areas of science to a given iDAC, and encourage users working in that field to use that iDAC. This could create a customer base for the iDAC, bring together like-minded experts, and effectively distribute the computing load across a network of iDACs. This might also enhance internal funding arguments for investment resources by arguing for synergies with local science goals and attracting international users and official endorsement.

%%%MLG: retained as a comment.
% In HEP experiments such as  BABAR various physics analysis groups (science collaborations in LSST ) were assigned to specific international centers as their primary computing and analysis facility, thereby distributing the computing load around the "network". People naturally tend to use the facility with available resources and cycles anyway of course. National groups received credit against their normal "operating common fund" contributions (equivalent to part of the LSST operations cost) based on their local computing contribution, and service to the full collaboration (equivalent to the full LSST data rights community). We have no such system in place for LSST though.

\subsection{Data Transfer}
Even with good networks the data transfer will not be trivial, and could be quite expensive. LSST is not currently set up to distribute data to multiple sites, i.e., there is no form of peer-to-peer sharing. The bandwidth at NCSA is adequate for receiving data and delivering {\tt Alerts} to brokers during the night; perhaps some day time bandwidth could be used to transfer data to iDACs. A full data release of images and catalogs does not have to transferred within a given day; if the correct agreements are in place with an iDAC, a full release could be transferred slowly as it is produced, and then made available to the iDACs users in whole on the official release day.
\input{xfercost}

\subsection{Compute {\it vs.} Storage Resources}
Data storage is a large cost to iDACs, and could be considered as an overhead relative to the amount of computational resources an iDAC can offer. If an iDAC is set up without a large compute capacity, the facility might be less useful to the science community than e.g., augmenting an existing DAC or iDAC to have more computational resources. It is conceivable that a partner institution may prefer to spend their money increasing the computational quotas available for a given collaboration or set of PIs, and it would be scientifically beneficial if this was possible at all DAC and iDACs. The notion of standard compute quotas and resource allocation committees to adjudicate on large proposals for substantial increases to computational allocations are described in \citeds{LPM-261}. Another way to approach a solution to this issue might be to have a \emph{Cloud}-based iDAC where a user or PI could buy nodes on the provider cloud to access the holdings put there by LSST.


\appendix
\section{References\label{sect:references}}
\renewcommand{\refname}{}
\bibliography{local,lsst,lsst-dm,refs,books,refs_ads}

\section{Acronyms}
\addtocounter{table}{-1}
\begin{longtable}{|l|p{0.8\textwidth}|}\hline
\textbf{Acronym} & \textbf{Description}  \\\hline

CPU&Central Processing Unit \\\hline
DAC&Data Access Center \\\hline
DAX&Data access services \\\hline
DPDD&Data Product Definition Document \\\hline
FTE&Full-Time Equivalent \\\hline
HEP&High-Energy Physics \\\hline
IT&Integration Test \\\hline
LDM&Light Data Management \\\hline
LPM&LSST Project Management (Document Handle) \\\hline
LSE&LSST Systems Engineering (Document Handle) \\\hline
LSST&Large Synoptic Survey Telescope \\\hline
NCSA&National Center for Supercomputing Applications \\\hline
PI&Principle Investigator \\\hline
RFC&Request for Comments \\\hline
TB&TeraByte \\\hline
US&United States \\\hline
USA&United States of America \\\hline
W&Watt; SI unit of power \\\hline
s&second; SI unit of time \\\hline
\end{longtable}
 % generated by the acronyms.csh (GaiaTools)

\input{checklist}
\end{document}



