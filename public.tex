\section{Public Data - the Open Data Framework}\label{sec:public}

Within the ODF LSST must support a large number number of users. Compared to the current model, perhaps 30$\%$ more professional astronomers. The total is probably comfortably bounder by a factor of two in any case.
Alex Szalay (ref) noted there were over a million unique IP addresses which hit the SDSS archive over a one year period.  Gaia saw 2K users accounts per hour on the catalogue release. These numbers suggest significant public interest beyond professionals, and this load must be supported by the DAC as well. There are several possible ways to handle the load.
\footnote{Gaia reports modest discontent due to user load \url{https://blogs.scientificamerican.com/observations/the-price-of-open-science/}},

Nationally, we could partner with Google or Amazon who will notionally host public data sets for free - so we could more selectively add users to the DACs and try to put most casual users to the public interface. It is not clear if this would then be the EPO interface, but that is worth considering. EPO would no longer have to select 10\% of data, but it would have a bigger job to deal with a full data release. Still, combining DM and EPO seamlessly with respect to data would be favorable.
The public data set would be some lite version for the catalog \secref{sec:lite}  and the HiPS type color images. No raw data files and potentially no advanced notebook type access. This may not even need a logon for the fast queries like $``$show me M31$''$ with LSST sources plotted. EPO queries would not include the source catalogue which is large ($10^{13}$ rows).
Potential national partners  could also host the object catalog e.g.

\begin{itemize}
 \item MAST at STScI
 \item DataLab at NOAO
 \item SAO
 \item US Naval Observatory
 \item NED at IPAC
 \item HEASARC
 \item CADC - Canadian Astronomical Data Centre - we work with them already in Data Management.
\end{itemize}

\subsection {International}
Internationally we could partner with a network of sites - it should be a network to allow peer to peer sharing of catalogs. Here we could provide the HIPS color images and again some version of the OBJECT catalogue. We would have to consider if the source catalogue should also be distributed to a smaller subset of centres who could cope with it.
Potential International Partners might be:

\begin{itemize}
\item In Europe we have a few centres that Astronomers expect to find sources at:
\begin{itemize}
    \item CDS  - Aladin and Vizier - this is a minimum for Europe
    \item ESAC - ESASky - European Space Agency
    \item  ASDC - Italian Data Center
    \item  GAVO MPA - German Astronomical Virtual Observatory - Max Plank Astrophysics
    \item  IoA - Cambridge
    \item  Edinburgh
\end{itemize}
\item In Asia
\begin{itemize}
    \item  NAOJ Astronomical Data Center, Tokyo
    \item  Chinese Academy of Sciences - perhaps National Space Science Center
\end{itemize}
\item In South America
\begin{itemize}
    \item We already have a Chilean DAC
    \item LIneA in Brazil - we already work with them
\end{itemize}
\end{itemize}

Having the lite catalog hosted at multiple locations where, and in formats which, astronomers would expect to find catalog information would reduce load on the US and Chilean DAC. It will also put the LSST data more readily in the hands of the astronomers and should accelerate science at least in the cases where the catalog is the prime source of information, for example, Galactic dynamics and other large statistical studies.
