

\section{Public Data - the Open Data Framework}\label{sec:public}


With public data be we face a potential huge number of users.
Alex Szalay at AAS noted there were over a million unique IP addresses which hit the SDSS archive over a one year period.  Gaia saw 2K users accounts per hour on the catalogue release - we are clearly not only serving the 10K professional astronomers in the world anymore.
This could be mitigated in two ways.


Nationally we could partner with Google or Amazon who will notionally host public data sets for free - so we could more selectively add users to the DACs and try to put most casual users to the public interface. It is not clear if this would then be the EPO interface but that is worth considering i.e. EPO would no longer have to select 10\% of data but it would have a much bigger job to deal with all of it perhaps.
The public data set would be some lite version for the catalog \secref{sec:lite}  and the HiPS type color images. No raw data files and potentially no advanced notebook type access. This may not even need a logon for the fast queries like show me M31 with LSST sources plotted. This would not include the source catalogue which is the big one ($10^{13}$ rows).
Potential national partners  could also host the object catalog e.g.

\begin{itemize}
 \item MAST at STScI
 \item DataLab at NOAO
 \item SAO
 \item US Naval Observatory
 \item NED at IPAC
 \item HEASARC
 \item CADC - Canadian Astronomical Data Centre - we work with them already in Data Management.
\end{itemize}

\subsection {International}
Internationally we could partner with s network of sites - it should be a network to allow peer to peer sharing of catalogs. Here we could provide the HIPS color images and again some version of the OBJECT catalogue. We would have to consider if the source catalogue should also be distributed to a smaller subset of centres who could cope with it.
Potential International Partners might be:

\begin{itemize}
\item In Europe we have a few centres that Astronomers expect to find sources at:
\begin{itemize}
    \item CDS  - Aladin and Vizier - this is a minimum for Europe
    \item ESAC - ESASky - European Space Agency
    \item  ASDC - Italian Data Center
    \item  GAVO MPA - German Astronomical Virtual Observatory - Max Plank Astrophysics
    \item  IoA - Cambridge
    \item  Edinburgh
\end{itemize}
\item In Asia
\begin{itemize}
    \item  NAOJ Astronomical Data Center, Tokyo
    \item  Chinese Academy of Sciences - perhaps National Space Science Center
\end{itemize}
\end{itemize}


Having the lite catalog hosted at multiple locations where, and in formats which, astronomers would expect to find catalog information would reduce load on the US and Chilean DAC.
 It will also put the LSST data more readily in the hands of the astronomers and should accelerate science at least in the cases where the catalog is the prime source of information e.g. galactic dynamics and other large statistical studies.
