

\section{Public Data - the Open Data Framework}\labe{sec:public}


WIth public data be we face a potential huge number of users.
Alex Szalay at AAS noted there were over a million unique IP addresses which hit the SDSS archive over a 1 year period.  Gaia saw 2K users accounts per hour on the catalogue release - we are clearly not only serving the 10K professional astronomers in the world anymore.
This could be mitigated in two ways.


Nationally we could partner with Google or Amazon who will host public data sets for free - so we could more selectively add users to the DACs and try to put most casual users to the public interface. It is not clear if this would then be the EPO interface but that is worth considering i.e. EPO would no longer have to select 10\% of data but it would have a much bigger job to deal with all of it perhaps.
The public data set would be some version for the catalogue and the HIPS type color images. No raw data files and potentially no advanced notebook type access. This may not even need a logon for the trivial stuff like show me M31 with LSST sources plotted.  So no source catalogue which is the big one (10^13 rows)
Potential national partners  could also host the object catalogue e.g.
* MAST at STScI
* DataLab at NOAO
* SAO
* Naval Observatory ?
* NED at IPAC ?
* HEASARC
* CADC - Canadian Astronomical Data Centre - we work with them already in DM.

International
Internationally we could partner with s network of sites - it should be a network to allow peer to peer sharing of catalogues. Here we could provide the HIPS color images and again some version of the OBJECT catalogue. We would have to consider if the source catalogue should also be distributed to e smaller subset of centres who could cope with it.
Potential International Partners
* In Europe we have a few centres that Astronomers expect to find sources at:
    * CDS  - Aladain and Vizier - this is a minimum for Europe
    * ESAC - ESASky - European Space Agency
    * ASDC - Itiallian Data Center
    * GAVO MPA - German Astronomical Virtual Observatory - Max Plank Astrophysics
    * IOA - Cambridge
    * Edinburogh
* In Asia
    * NAOJ JADC- Tokyo - Given the HSC connection and the fact that Kazuhiro Sekiguchi approached at AAS with interest in joining LSST we shoudl talk to them
    * Some place in Australia - though they have been non committal
    * China ?
P

