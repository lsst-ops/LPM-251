\subsubsection{Transfer cost use case}
If we take the final number from the key numbers page \footnote{\url{https://confluence.lsstcorp.org/display/LKB/LSST+Key+Numbers}} we could consider DR1 as about 6 PB (10\% of the final size).

We would have at least  two ways to transfer this : via the network, via physical devices.

A network transfer at 10Gbps of 6 PB would take $8 * 6 \times 10^{12} \ 10^7 = 4.8 \times 10^{6}~seconds $ or about 55 days\footnote{ day = 86400}.
Many institutes have 100 Gbps connections so this should be an upper limit and a transfer should be order one week. If we had a peer to peer network this may go down somewhat and we may be able to support it from NCSA.

Alternatively we could host the data on Amazon or Google and let people download it from there - they would have more capacity.
Storage on the cloud  for public data would be theoretically free - download (egress)  would  cost.
Transfer to another cloud \footnote{\url{https://cloud.google.com/storage/pricingi\#network-pricing}}  or
 a Content Delivery Network (CDN)\footnote{\url{https://cloud.google.com/cdn/pricing}}
 end up costing  about a cent a GB which for an open science project and at our volume  should  be negotiable.  At one cent a transfer would cost
  $\sim \$0.01 * 6 \times 10^{12} \ 10^6 = \$60K$.
So we would need some special deal to make that work.

For physical devices, today apparently we could get a device like Petarack \url{https://www.aberdeeninc.com/petarack/} for \$300K.
This could be be a partner data center cost as well as shipping it. Transfer of data on to this would be about the same as the network rate above so 7 days. SneakerNet \cite{2002cs........8011G} may still be  cost effective in the LSST era, which is predicted in the a paper.


