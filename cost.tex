\section{Cost Impacts}\label{sec:costs}


As previously mentioned, standing up and maintaining multiple iDACs comes at a significant cost impact to both the LSST Project and the partner institutions. Minimizing these costs -- or at least maximizing the amount of science they enable -- should be at the forefront of all considerations concerning partner iDACs, such as the following propositions.

\subsection{Maximizing Profits with Science-Driven iDACs}
There are two main cost impacts of iDACs being set up outside of the US and Chilean DACs: the positive impact is that some computational load may be taken off of these existing DACs, but the negative impact is the level of support required from the LSST Project in order to get them set up and running. This negative impact could be mitigated by ensuring that science productivity is maximized as a result of this extended effort. One way to do this might be to associate specific areas of science to a given iDAC, and encourage users working in that field to use that iDAC. This could create a customer base for the iDAC, bring together like-minded experts, and effectively distribute the computing load across a network of iDACs. This might also enhance internal funding arguments for investment resources by arguing for synergies with local science goals and attracting international users and official endorsement.


\subsection{Data Transfer}\label{sec:xfer}
Even with good networks the data transfer will not be trivial, and could be quite expensive. LSST is not currently set up to distribute data to multiple sites, i.e., there is no form of peer-to-peer sharing. The bandwidth at NCSA is adequate for receiving data and delivering {\tt Alerts} to brokers during the night; perhaps some day time bandwidth could be used to transfer data to iDACs. A full data release of images and catalogs does not have to transferred within a given day; if the correct agreements are in place with an iDAC, a full release could be transferred slowly as it is produced, and then made available to the iDACs users in whole on the official release day.
\subsubsection{Transfer cost use case \label{sec:xfercost}} 
If we take the final number from the key numbers page \footnote{\url{https://confluence.lsstcorp.org/display/LKB/LSST+Key+Numbers}} we could consider DR1 as about 6 PB (10\% of the final size).

We would have at least  two ways to transfer this : via the network, via physical devices.

A network transfer at 10Gbps of 6 PB would take $8 * 6 \times 10^{12} \ 10^7 = 4.8 \times 10^{6}~seconds $ or about 55 days\footnote{ day = 86400}.
Many institutes have 100 Gbps connections so this should be an upper limit and a transfer should be order one week. If we had a peer to peer network this may go down somewhat and we may be able to support it from NCSA.

Alternatively we could host the data on Amazon or Google and let people download it from there - they would have more capacity.
Storage on the cloud  for public data would be theoretically free - download (egress)  would  cost.
Transfer to another cloud \footnote{\url{https://cloud.google.com/storage/pricingi\#network-pricing}}  or
 a Content Delivery Network (CDN)\footnote{\url{https://cloud.google.com/cdn/pricing}}
 end up costing  about a cent a GB which for an open science project and at our volume  should  be negotiable.  At one cent a transfer would cost
  $\sim \$0.01 * 6 \times 10^{12} \ 10^6 = \$60K$.

For physical devices, today apparently we could get a device like Petarack \url{https://www.aberdeeninc.com/petarack/} for \$300K.
Theoretically we could get this cheaper though this is close to the drive price,
Tape may also be a possibility especially if Sony/IBM commercialize high density tape with >300TB per cartridge\footnote{\url{https://newatlas.com/sony-ibm-magnetic-tape-density-record/50743/}}. A current 6TB cartridge is about \$30, so enough tapes for 6PB would cost about  30K. If the density increased this could come down significantly.
This could be be a partner data center cost as well as shipping it. Transfer of data on to this would be about the same as the network rate above so 7 days. SneakerNet \cite{2002cs........8011G} may still be  cost effective in the LSST era, which is predicted in the a paper.




\subsection{Compute {\it vs.} Storage Resources} \label{sec:cvs}
Data storage is a large cost to iDACs, and could be considered as an overhead relative to the amount of computational resources an iDAC can offer. If an iDAC is set up without a large compute capacity, the facility might be less useful to the science community than e.g., augmenting an existing DAC or iDAC to have more computational resources. It is conceivable that a partner institution may prefer to spend their money increasing the computational quotas available for a given collaboration or set of PIs, and it would be scientifically beneficial if this was possible at all DAC and iDACs. The notion of standard compute quotas and resource allocation committees to adjudicate on large proposals for substantial increases to computational allocations are described in \citeds{LPM-261}. Another way to approach a solution to this issue might be to have a \emph{Cloud}-based iDAC where a user or PI could buy nodes on the provider cloud to access the holdings put there by LSST.

