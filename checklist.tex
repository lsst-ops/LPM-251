\newlist{todolist}{itemize}{2}
\setlist[todolist]{label=$\square$}

\section{IDAC Proposal Checklist}\label{sec:checklist}

There is a spectrum of possibilities for an IDAC's scope, from just hosting the Object lite database, to serving full copies of the current data release and prompt products database.
Here an attempt is made to have a set of check lists  that can be used to look at an IDAC.
\secref{sec:fullDAC} covers a full capability IDAC, while  \secref{sec:liteDAC} gives the criteria for a minimum capability IDAC.
There will undoubtedly be proposals in between. There are some criteria that any IDAC must meet, in order to comply with Rubin Observatory data policy. These are given in \secref{anyDAC}.

\subsection{Any DAC} \label{sec:anyDAC}
\begin{todolist}
\item Authentication/Authorization system  inline with Rubin Observatory Access
\item Agreement to make broadly accessible to all Data Rights holders
\end{todolist}

\subsection{Lite IDAC} \label{sec:liteDAC}
All criteria in \secref{sec:anyDAC}, and then, in addition:
\begin{todolist}
\item Database system capable of handling $4^{10}$ rows.
\item IVOA TAP interface, MyDB and Table Upload, CAOM support.
\item About 500TB of disk for catalogs + MyDBs.
\item Professional support staff (min 0.25 FTE)
\item Sufficient connectivity to support users
\end{todolist}

\subsection{Full DAC} \label{sec:fullDAC}
All criteria in \secref{sec:anyDAC}, and then, in addition:
\begin{todolist}
\item Staff (about 5 FTE) to handle major hardware installation
\item Agreement to stand up standard Science Platform (Puppet/Kubernetes etc.)
\item Database system capable of handling all catalogs (or Qserv) with IVOA  interfaces.
\item Understanding of sizing model in \citeds{DMTN-135} - sizing model and a cost model for the IDAC.
\item "Commitment to fund the IDAC through the LSST operations period, FY22-FY34 (probably  $>\$6M/year$ based on hardware cost model and labor plan).".
\item Sufficient connectivity to support data transfer in and user access out at least 20Gbps of free bandwidth.
\end{todolist}
